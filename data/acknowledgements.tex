% !TeX root = ../main.tex

% 盲审时需删掉,在 main.tex 注释掉即可

\begin{acknowledgements}
「反正你还年轻,人生有无限可能。」似乎的确经常从这里或那里看到或听到这样的话。
但我已诞生于此世间足足有四分之一个世纪,同样也几乎走完了人生道路中最重要的四分之一。
时至今日,原本某些柔软的东西也像胡茬一样开始变得硬邦邦的,再也无法逆转。

自三年前进入中国传媒大学读研以来,我的想法与人生目标已发生了许多变化。
我的状态也如同状态机一般不断地从一种状态过渡到另一种状态,可惜其并非确定有限型的。
我的初态与自己的期望相去甚远,但在此期间,却很庆幸地遇到了许多无法预测的可爱的人和事,因此,我决定把它叫做非确定无限状态机。

「假如当时选择了别的道路,会过着不同的大学生活吗?」我不止一次又一次地询问自己,并幻想平行世界中自己的现状。
人生就像是玩文字冒险游戏,做着一个又一个的选择,最后迎来自己所决定的结局。
但与确定有限状态机「确定输入」返回「确定结果」不同,即便同样的初态与输入,时机不同,我也未必拥有相同的结果。
继续倒退,在灰暗的中学时代,我也曾幻想在进入大学后过上如「四叠半神话」中所言的玫瑰色校园生活,与明石相遇、在社团挥洒汗水、精进学业、锻炼身体,未来成为社会栋梁。
但世事难料,我机缘巧合地到了我的母校就读计算机专业,并拖着步伐踩着绊脚石把时光花费在每一条岔路上,最终成为奇形怪状的自己。
而我又抱着类似的希冀与偏执,打了几次挺,来到了如今的母校,并一定程度上再次重复了幻想与放弃的过程。
只是在我以为会望到尽头的研究生生活中,与诸多未曾设想的人与事不期而遇,并走向了人生剧本前几幕中一个还算不错的小结局。

在写代码还是论文时,我似乎总有些洁癖,总是试图寻找更完美的解决方案与排版格式,并在其中蹉跎了大量时间。
也许在几次反复后,项目终究可以收敛于目前可见的最佳方式。
但人生却不会像 ADV 一样可以存档读档,二次修正,从来没有完美的人生与故事,不过正如阿德勒所言,人的可爱之处便在于接受不完美的自己。
过去所有的人与事就如同正向动力学中的骨骼解算,连续的关节依次计算出了如今的自己。
仅仅某个节点的偏差,便可能使端点大相径庭。
我也想不出有限骨骼抵达目前结局的更好解算方式,因此我很感谢在我人生轻小说剧本中出场的关键人物们与此期间的变量。

首先衷心感谢我的引路人XXX老师,这篇论文的顺利完成,离不开她的悉心指导和鼓励。
在实验室的学习生涯和平时的生活中,也从X老师这里学到了很多。
其次则要感谢我的导师XXX老师,平易近人的X老师给了我更多的自由空间。
也十分感谢XXX老师,在游戏策划课上以及实验室例会等,从X老师这里开阔了视野与眼界。
还有在研一研二学习课程时,学院老师们给予的指导。
他们的言传身教将使我终生受益。而我想,这大概也是中传生涯这块骨骼解算的起点。

也感谢实验室的同窗伙伴们X sXX、XXX、XXX、XX、XXX和XXX,和他们共同度过的实验室生活是我难以忘怀的宝贵记忆。
感谢XXX师兄解答了很多写论文时规范经验的问题,感谢XXX、XXX师妹们帮忙校验。
感谢实验室所提供的学习平台与轻松愉快的氛围,也感谢 CBD 中的每一位同学。
感谢过去与现在的室友们,以及同班同学们在校园所交织的故事。
感谢腾讯与支付宝实习期间遇到的同事们,从他们身上我学到了不少开发经验。
感谢开源社区与开源项目们,让我得以快速完成开发工作,并从中学到了许多代码、项目的实践与思考。
感谢相关课题与参考文献的研究学者们,感谢 \LaTeX 与 \thuthesis,让我得以更好地完成论文的撰写与排版工作。

我也想对父母道声谢,在家写论文期间他们为我营造了一个衣食无忧,安静的学习环境,也是他们的支持和鼓励推动着我成为现在的自己。

如今我又将抱着新的期冀启程,幻想玫瑰色的生活,续写有限又无限的可能,直至我的状态机抵达终态。
我不知道到那时骨骼解算的终点在何处,但我想,当我的剧本落下帷幕时,我一定还会回忆并感谢此间登场的变量与关键人物,你们。
  
\end{acknowledgements}
  