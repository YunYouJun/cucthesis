% !TeX root = ../main.tex

\chapter{绪论}

我习惯在每章节加一段引入的话,譬如面对XXX带来的挑战,本文提出了一种XXX方法。
本章节就其研究背景、意义、现状以及本文研究内容与组织结构进行简要介绍。


\section{研究背景及意义}

\subsection{研究背景}

一堆乱七八糟的研究背景,加一点宏大叙事。简称懂的都懂。

中国互联网络信息中心(CNNIC) 2021 年 8 月 27 日在京发布的第48次《中国互联网络发展状况统计报告》显示,
% 较2020年12月增长2175万,
截至2021年6月,我国网民规模达10.11亿,
互联网普及率达 71.6\%。
十亿用户接入互联网,形成了全球最为庞大、生机勃勃的数字社会。\cite{cnnic48report}
其中我国网络游戏用户规模达 5.09 亿,占网民整体的 50.4\%。



游戏产业的不断发展,使得游戏的经济属性和文化价值显现得愈加明显,越来越多的人接受了这种集休闲娱乐与文化传播为一体的新兴媒介。
据 Newzoo 2021 全球游戏市场报告\cite{newzoo2021report}表明(如图~\ref{fig:newzoo-player-forecast}),
\begin{figure}[htb]
  \centering
  % 这里可以控制图片宽度比例
  \includegraphics[width=0.8\linewidth]{steins-gate-game-screenshot.jpg}
  \caption{2015-2024年全球玩家数\cite{newzoo2021report}}
  \label{fig:newzoo-player-forecast}
\end{figure}
2021 年全球游戏市场收获近 1758 亿美元,
全球游戏玩家数量将持续增长,到 2023 年将超过三十亿,市场收入也在随之不断增长。
全球三十亿玩家中移动玩家数量约为 28 亿,PC 及主机玩家分别为 14 亿与 9 亿。

...

\subsection{研究意义}

写写研究意义。

% https://www.zhihu.com/question/393670234
关于这个事,我简单说两句,至于我的身份,你明白就行。
总而言之,这个事呢,现在就是这个情况,具体的呢,大家也都看得到,我因为这个身份上的问题,也得出来说那么几句,可能,你听的不是很明白,但是意思就是那么个意思。
我的身份呢,不知道的你也不用去猜,这种事情见得多了,我只想说懂得都懂,不懂的我也不多解释,毕竟自己知道就好,细细品吧。
你们也别来问我怎么了,利益牵扯太大,说了对你我都没好处,当不知道就行了,其余的我只能说这里面水很深,牵扯到很多东西。
详细情况你们自己是很难找的,网上大部分已经删除干净了,所以我只能说懂得都懂。
懂的人已经基本都获利上岸什么的了,不懂的人永远不懂,关键懂的人都是自己悟的,你也不知道谁是懂的人也没法请教,大家都藏着掖着生怕别人知道自己懂事,懂了就能收割不懂的,你甚至都不知道自己不懂。
只是在有些时候,某些人对某些事情不懂装懂,还以为别人不懂。
其实自己才是不懂的,别人懂的够多了,不仅懂,还懂的超越了这个范围,但是某些不懂的人让这个懂的人完全教不懂,所以不懂的人永远不懂,只能不懂装懂。
别人说懂的都懂,只要点点头就行了,其实你懂的我也懂,谁让我们都懂呢,不懂的话也没必要装懂,毕竟里面牵扣扯到很多懂不了的事。
这种事懂的人也没必要访出来,不懂的人看见又来问七问八,最后跟他说了他也不一定能懂,就算懂了以后也对他不好,毕竟懂的太多了不是好事。
所以大家最好是不懂就不要去了解,懂太多不好。

\section{国内外研究现状}

可以分几节来写。
对了一般句子末尾最好不要单独一个字另起一行。

\subsection{论文主要部分的写法}

研究生学位论文撰写,除表达形式上需要符合一定的格式要求外,内容方面上也要遵循一些共性原则。

通常研究生学位论文只能有一个主题(不能是几块工作拼凑在一起),该主题应针对某学科领域中的一个具体问题展开深入、系统的研究,并得出有价值的研究结论。
学位论文的研究主题切忌过大,例如,“中国国有企业改制问题研究”这样的研究主题过大,因为“国企改制”涉及的问题范围太广,很难在一本研究生学位论文中完全研究透彻。

\subsection{论文的语言及表述}

除国际研究生外,学位论文一律须用汉语书写。
学位论文应当用规范汉字进行撰写,除古汉语研究中涉及的古文字和参考文献中引用的外文文献之外,均采用简体汉字撰写。

国际研究生一般应以中文或英文书写学位论文,格式要求同上。
论文须用中文封面。

研究生学位论文是学术作品,因此其表述要严谨简明,重点突出,专业常识应简写或不写,做到立论正确、数据可靠、说明透彻、推理严谨、文字凝练、层次分明,避免使用文学性质的或带感情色彩的非学术性语言。

论文中如出现一个非通用性的新名词、新术语或新概念,需随即解释清楚。

\subsection{论文题目的写法}

论文题目应简明扼要地反映论文工作的主要内容,力求精炼、准确,切忌笼统。
论文题目是对研究对象的准确、具体描述,一般要在一定程度上体现研究结论,因此,论文题目不仅应告诉读者这本论文研究了什么问题,更要告诉读者这个研究得出的结论。
例如:“在事实与虚构之间:梅乐、卡彭特、沃尔夫的新闻观”就比“三个美国作家的新闻观研究”更专业、更准确。

\section{本文研究内容}

总结本文研究内容。
本文的主要目的是……

一篇学位论文的引言大致包含如下几个部分:
1、问题的提出;
2、选题背 景及意义;
3、文献综述;
4、研究方法;
5、论文结构安排。
\begin{itemize}
  \item 问题的提出:要清晰地阐述所要研究的问题“是什么”。
  \item 选题背景及意义:论述清楚为什么选择这个题目来研究,即阐述该研究对学科发展的贡献、对国计民生的理论与现实意义等。
  \item 文献综述:对本研究主题范围内的文献进行详尽的综合述评,“述”的同时一定要有“评”,指出现有研究状态,仍存在哪些尚待解决的问题,讲出自己的研究有哪些探索性内容。
  \item 研究方法:讲清论文所使用的学术研究方法。
  \item 论文结构安排:介绍本论文的写作结构安排。
\end{itemize}

\section{本文组织结构}

本文通过以下六个章节详细论述该研究课题,结构内容如下:

第一章:绪论。本章讨论了XXX以及它所代表的XX方式的意义与可在经济、社会价值上起到的作用,
并回顾国内外现状,明确了本文研究内容与组织结构。

第二章:相关技术概述。本章对后续研究所要使用到的相关技术及概念进行了阐述。文中依次介绍了XXX、XXX、XX与XX等技术的基础原理与作用。...

第三章:XXX。...

第四章:XXX。...

第五章:系统设计与应用。本章从软件工程角度概述了系统的整体设计架构与设计方法,具体介绍了相关开发实现方案与特色功能,将三四章研究内容串联并应用以完成系统开发。
最后对系统进行了展示与测试评估。

第六章:总结与展望。本文目标内容已基本顺利完成,但其研究成果仍然存在许多改进空间。该部分对研究内容与系统成品进行回顾,总结了主旨与存在的问题。
通过思考相应改进措施,对未来研究内容与方向进行展望。
